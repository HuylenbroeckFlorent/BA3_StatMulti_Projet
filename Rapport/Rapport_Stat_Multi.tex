\documentclass{article}

% Use utf-8 encoding for foreign characters
\usepackage[utf8]{inputenc}

% Setup for fullpage use
\usepackage{fullpage}
\usepackage{graphicx}
\usepackage{hyperref}
\usepackage[english]{babel}
\usepackage[final]{pdfpages}
\usepackage{times}
\usepackage{gensymb}
\usepackage{color}
\usepackage{array}
\usepackage{textcomp}
\usepackage{subfig}
\usepackage{amsmath}
\usepackage{titlesec}

\includepdfset{pages = -, pagecommand = {}, scale = 0.9}
\setcounter{secnumdepth}{4}
\newcommand{\placeholder}[1]{{\noindent \color{red}[ #1 ]}}
\newtheorem{Rem}{Remarque}
\titleformat{\paragraph}
{\normalfont\normalsize\bfseries}{\theparagraph}{1em}{}
\titlespacing*{\paragraph}
{0pt}{3.25ex plus 1ex minus .2ex}{1.5ex plus .2ex}



\begin{document}


\title{
{\Huge Rapport du Projet\\
Statistique Multidimensionnelle\\
\smallskip
\author{
\textbf{Groupe Info n\degree 1}\\
JOSSE Thomas, HUYLENBROECK Florent, DELFOSSE Charly\\
}}
}
\date{Année Académique  2018-2019\\
 Bachelier en Sciences Informatiques
\\
\vspace{1cm}
Faculté des Sciences, UMons}



\maketitle            % typeset the title of the contribution
\bigskip
\begin{center} \today \end{center}
\begin{abstract}
Ce rapport est écrit dans le cadre du cours de \texttt{Statistique Multidimensionnelle} dispensé par M. \emph{Michel VOUÉ}. Ce projet consistait en l'application de l'\texttt{Analyse en Composantes Principales} vues au cours sur un cas concret, ainsi que la découverte d'une technique d'analyse non abordée au cours à savoir, l'\texttt{Analyse Discriminante Linéaire} ou ADL.
\end{abstract}
\newpage
\tableofcontents
\newpage

\section{Question 1 : ACP sur un cas concret}
\section{Question 2 : \texttt{Analyse Discriminante Linéaire}}
\end{document}